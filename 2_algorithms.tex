\documentclass{article}
\usepackage[utf8]{inputenc}

\usepackage[ruled]{algorithm2e}

\title{2 -- Algorithms}

\begin{document}
\maketitle

What's an excellent way to communicate an algorithm? It sounds
pretty challenging to do with just words. Below we present the
\texttt{algorithm2e} package, which allows you to write algorithms
and pseudocode in LaTeX.

\section{Bisection}

Remember that the Bisection method takes intervals $[a, b]$ and
chops them in half until the root is found.

\begin{center}
	\begin{minipage}{0.5\linewidth} % Adjust the minipage width to accomodate for the length of algorithm lines
		\begin{algorithm}[H]
			\KwIn{$f, a, b, M, \delta, \epsilon$}  % Algorithm inputs
			\KwResult{$x$ such that $f(x) \approx 0$} % Algorithm outputs/results

      $u \gets f(a)$

      $v \gets f(b)$

      $e \gets b - a$

        \For{$k=1$ \KwTo $M$}{
          $e \gets e/2$

          $c \gets a + e$

          $w \gets f(c)$

          \If{$|e| < \delta$ or $|w| < \epsilon$}{
            {\bf stop}
          }
          \If{sign($w$) $\neq$ sign($u$)}{
            $b \gets c$
            
            $v \gets w$
          }
          \Else{
            $a \gets c$

            $u \gets w$
          }
        }

			{\bf return} $c$
			\caption{\texttt{Bisection}} % Algorithm name
			\label{alg:bisection}   % optional label to refer to
		\end{algorithm}
	\end{minipage}
\end{center}

\section{Newton}

Remember that Newton's method requires the derivative of the function.
It rides the tangent line at the current point to the $x$-axis, and
then repeats.

\begin{center}
	\begin{minipage}{0.5\linewidth} % Adjust the minipage width to accomodate for the length of algorithm lines
		\begin{algorithm}[H]
      \KwIn{$f, f', x_0, M, \delta, \epsilon$}
      \KwResult{$x$ such that $f(x) \approx 0$}

      $v \gets f(x_0)$

      \If{$|v| < \epsilon$}{
        {\bf stop}
      }

      \For{$k=1$ \KwTo $M$}{
        $x_1 \gets x_0 - v/f'(x_0)$

        $v \gets f(x_1)$

        \If{$|x_1 - x_0| < \delta$ or $|v| < \epsilon$}{
          {\bf stop}
        }

        $x_0 \gets x_1$
      }

      {\bf return} $x_1$


			\caption{\texttt{Newton's}} % Algorithm name
			\label{alg:newton}   % optional label to refer to
		\end{algorithm}
	\end{minipage}
\end{center}

\section{Secant}

Similar to Newton's method, Secant method uses the secant line
and finds the root of that line. Secant method doesn't use the derivative,
which is cheaper, but it's not as fast as Newton's method.

\begin{center}
	\begin{minipage}{0.5\linewidth} % Adjust the minipage width to accomodate for the length of algorithm lines
		\begin{algorithm}[H]
      \KwIn{$f, a, b, M, \delta, \epsilon$}

      $fa \gets f(a); fb \gets f(b)$

      \For{$k = 2$ \KwTo $M$}{
        \If{$|fa| < |fb|$}{
          $a \leftrightarrow b; fa \leftrightarrow fb$
        }
        $s \gets (b - a)/(fb - fa)$

        $b \gets a; fb \gets fa$

        $a \gets a - fa * s$

        $fa \gets f(a)$

        \If{$|fa| < \epsilon$ or $|b - a| < \delta$}{
          {\bf stop}
        }
      }
      {\bf return} $a$
      
			\caption{\texttt{Secant}} % Algorithm name
			\label{alg:secant}   % optional label to refer to
		\end{algorithm}
	\end{minipage}
\end{center}
\end{document}