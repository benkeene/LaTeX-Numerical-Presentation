\documentclass{article}
\usepackage[utf8]{inputenc}

\usepackage{minted}

\title{3 -- Displaying Source Code Well}
\begin{document}

\maketitle

Suppose you want to include code in your document. Notice that
I've used a teletype font at some points in this document.
For example, it looks like \texttt{this}. While you might think
about using it to write code, it would become near impossible
to do this for a lot of code.

\section{Bisection}

Let's show some code. We'll only look at the Bisection method, but you
are free to recreate this for Newton's and Secant methods.

\subsection{Code Snippet (Verbatim)}

A simple to use environment to parse text literally as it's written is \texttt{verbatim}.

\begin{verbatim}
  def bisection(f, a, b, M, delta, epsilon):
    u = f(a)
    v = f(b)
    e = b - a

    for k in range(1, M + 1):
        e = e/2
        c = a + e
        w = f(c)

        if abs(e) < delta or abs(w) < epsilon:
            break

        if np.sign(w) != np.sign(u):
            b = c
            v = w
        else:
            a = c
            u = w

    return c
\end{verbatim}

It looks good, the tabs are rendered correctly. But this would be a bummer
if we've already written this code in another file.

\subsection{Code Snippet (Inline with minted)}

The next three subsections discuss the use of the package \texttt{minted}.
From henceforth you must add the \texttt{-shell-escape} flag when you compile.

What if we just wanted some short code inline? The \texttt{mintinline} command
shows code inline. This could be used in a footnote.

\mintinline{python}{def hello_world(): print("Hello, world!")}


\subsection{Code Snippet (Minted)}

What about bigger blocks of code? What other cool arguments can we pass?
\begin{itemize}
    \item \texttt{linenos} - print line numbers
    \item \texttt{autogobble} - remove leading whitespace, relative tabbing
    \item \texttt{breaklines} - break long lines instead of running off
    \item \texttt{mathescape} - use \texttt{\$} to enter math mode
\end{itemize}

\begin{minted}[linenos, autogobble, breaklines=true, mathescape=True]{python}
# test $f(x)$ $\sum_{n=1}^{\infty} \frac{1}{n^2}$
  def bisection(f, a, b, M, delta, epsilon):
    u = f(a)
    v = f(b)
    e = b - a

    for k in range(1, M + 1):
        e = e/2
        c = a + e
        w = f(c)

        if abs(e) < delta or abs(w) < epsilon:
            break

        if np.sign(w) != np.sign(u):
            b = c
            v = w
        else:
            a = c
            u = w

    return c
\end{minted}
\subsection{Code Snippet (Minted with external file)}

What if we already wrote the code in a file and we don't want to
rewrite it. Then we can use the \texttt{inputminted} command.
We're still using the \texttt{linenos}, \texttt{autogobble}, \texttt{breaklines=true},
and \texttt{mathescape=True} arguments.

We have two new arguments, \texttt{firstline} and \texttt{lastline}.

\inputminted[linenos, autogobble, breaklines=true, mathescape=true, firstline=6, lastline=41]{python}{./algorithms/algorithms.py}

\end{document}